%----------------------------------------------------------------------------------------
% Essay
%----------------------------------------------------------------------------------------
\setcounter{page}{1} % Sets counter of page to 1

\section*{1. Week}

\paragraph{Set Up and Project Planning} After onboarding with Maria, I set up my software environment, using the programming language Python and a private Github repository. I planned the project, with the subgoals to finish model implementation within the first two weeks and replicating model behavior to visual inputs such as the medial axis effect \cite{zhaoping_understanding_2014} (including puffer time to debug the model) within another two weeks until the end of October. In my remaining time of approximate 4 weeks in November, I am planning to modify the replicated model and test it on new visual inputs. The remaining two weeks in December allow me to prepare my seminar presentation and finish the lab report. I brainstormed ideas for extensions of the V1 model with Vlad, and we came up with the idea to try finding new visual search asymmetries with the V1 model and test those predictions experimentally.  

\begin{figure}[htbp!]
	\centering
	\includegraphics[width=0.85\linewidth]{figure/calibration_inputs.png}
	\caption{Visual inputs replicated from Fig. 4.18 \cite{zhaoping_understanding_2014}}
	\label{fig:inputs}
\end{figure}

\begin{wrapfigure}{rt}{0.4\textwidth}
	\vspace{-15pt} % adjust vertical spacing
	\centering
	\includegraphics[width=0.38\textwidth]{figure/tuning_curve.png}
	\caption{Implemented tuning curve decays exponentially and drops to $0$ for $|x| \geq \frac{\pi}{6} \approx 0.52$ as expected.}
	\label{fig:tc}
	\vspace{-5pt} % tighten bottom space
\end{wrapfigure}
\paragraph{V1 Model Implementation} I started by implementing visual inputs, see Fig.\ \ref{fig:inputs}. 
The visual input to the model consists of bars with contrast $\hat{I}_{i\gamma}$ oriented at angles $\gamma$ and is filtered through the tuning curve $\phi$ of the model neuron at location $i$ with preferred orientation $\theta$, such that $I_{i\theta} = \hat{I}_{i\gamma} \cdot \phi(\theta - \gamma)$. The implemented orientation tuning curve \cite{zhaoping_understanding_2014} 
\begin{equation}
	\phi(\theta - \gamma) = \begin{cases}
		0, & \text{if } |\theta-\gamma| \geq \frac{\pi}{6} \\
		\exp(-\frac{|\theta - \gamma|}{\pi / 8}),  & \text{otherwise}
	\end{cases}
\end{equation}
is shown in Fig.\ \ref{fig:tc}. 
\noindent
I further implemented a toy model 
\begin{equation} \label{eq:toy_df}
	\dot x_{i\theta} = - \alpha_x \cdot  x_{i\theta} + I_{i\theta}
\end{equation}
with $K = 12$ neurons per hypercolumn and $\alpha_x = 1$ to verify my implementation of the forward Euler method for numerical simulations
\begin{equation}
	\Delta x_{i\theta} \approx \dot x_{i\theta} \Delta t = (- \alpha_x \cdot  x_{i\theta} + I_{i\theta}) \Delta t.
\end{equation}
The analytic solution (for constant input solvable by separating variables to obtain the homogeneous solution of Eq.\ \ref{eq:toy_df}, and then varying the constants to obtain the full solution)
\begin{equation}
	x_{i\theta}(t) = \frac{I_{i\theta}}{\alpha} \left[1 - \exp(-\alpha_x t) \right]
\end{equation}
for $x_{i\theta}(t = 0) = 0$ was compared to the simulated trajectory using $\Delta t = 10^{-3}$, see Fig.\ \ref{fig:toy_model}.

\begin{figure}[htbp!]
	\centering
	\includegraphics[width=0.8\linewidth]{figure/toy_model.png}
	\caption{Response of toy model neurons of one hypercolumn to a bar of $90^{\circ}$ orientation. As expected, an exponential decay to a resting state - the filtered input - is observed. Neurons tuned closer to the orientation of the input bar show higher resting state. Simulated (colored) and analytical (grey) trajectories overlap precisely.}
	\label{fig:toy_model}
\end{figure}

In the end, I started to read and understand the stability analysis of the V1 model (pp.\ 285 onwards \cite{zhaoping_understanding_2014}). I also started to understand the noise term and learn more about stochastic differential equations, especially the Ornstein–Uhlenbeck (OU) process to model temporal correlations. 

\paragraph{Goals for coming week}
\begin{itemize}
    \item Understand OU process and implement noise term (Euler-Maruyama method)
    \item finish reading stability analysis, understand connectivity parametrization and implement connections
    \item implement remaining model parts, i.e.\ activation functions and normalization term
    \item verify full V1 model by replicating Fig.\ 5.18 (pp. 224) and 5.21 (pp. 229) \cite{zhaoping_understanding_2014}
\end{itemize}

\section*{2. Week}

\paragraph{Review of Goals from last week}

\begin{itemize}
	\item Implemented noise model as noise pulses of average height (i.e.\ Root-Mean-Square) $\sigma_{noise}$ and average temporal width $\tau_{noise}$, such that the noise input to neuron $i$ with preferred orientation $\theta$ is distributed as
		\begin{align*}
			\Delta t_{i \theta} &\sim \text{Exp}(\frac{1}{\tau_{noise}}) \\
			I_{noise}(i, \theta) &\sim \mathcal{N}(0, \sigma_{noise}^2)
		\end{align*}
	\item Understood mathematical parametrization and implemented connections, see Fig. \ref{fig:J}, \ref{fig:W} and \ref{fig:psi}
	\item Implemented remaining model parts (normalization and activation functions, see Fig.\ \ref{fig:g}) and increased efficiency of simulating full model using parallelization and compilation tricks
	\item Replicated Fig.\ 5.18 \cite{zhaoping_understanding_2014}, see Fig.\ \ref{fig:5.18} 
	\item Replicated Fig.\ 5.21 \cite{zhaoping_understanding_2014}, see Fig.\  \ref{fig:5.21}
	\item Started debugging full model: line-by-line review and second implementation 
	\item Read and summarized Chapter 5.8.1.1 - 5.8.2.3 of \cite{zhaoping_understanding_2014}
\end{itemize}

\begin{figure}[htbp!]
	\centering
	\caption{Excitatory connections $J_{i\theta,j\theta'}$}
	\includegraphics[width=\linewidth]{figure/J.png}
	\label{fig:J}
\end{figure}

\begin{figure}[htbp!]
	\centering
	\caption{Inhibitory connections $W_{i\theta,j\theta'}$}
	\includegraphics[width=\linewidth]{figure/W.png}
	\label{fig:W}
\end{figure}

\begin{figure}[htbp!]
    \centering
	% Main caption
    \caption{Activation functions $g_x$ and $g_y$}
    % Subfigure 1
    \begin{subfigure}[b]{0.47\textwidth}
        \centering
        \includegraphics[width=\textwidth]{figure/g_x.png}
        % \caption{Caption for $g_x$}
        \label{fig:gx}
    \end{subfigure}
    \hfill
    % Subfigure 2
    \begin{subfigure}[b]{0.47\textwidth}
        \centering
        \includegraphics[width=\textwidth]{figure/g_y.png}
        % \caption{Caption for $g_y$}
        \label{fig:gy}
    \end{subfigure}
    \label{fig:g}
\end{figure}

\begin{figure}[htbp!]
	\centering
	\includegraphics[width=0.9\linewidth]{figure/psi.png}
	\caption{Spread of inhibition within one hypercolumn $\psi$}
	\label{fig:psi}
\end{figure}

\begin{figure}[htbp!]
	\centering
	\includegraphics[width=\linewidth]{figure/Fig. 5.18.png}
	\caption{Qualitative replication of contextual influences from Fig.\ 5.18 \cite{zhaoping_understanding_2014}. Even though harder to see by eye, C shows smaller response to the center bar than D, and F shows smaller response to the center bar than E. Absolute differences in responces to the center bar seem to look smaller than in the reference figure. From E - F, colinear facilitation seems to increase for smaller bar contrast less, and for higher bar contrast more than in the reference figure. The random background in C seems to supress the reponse less than in the reference figure, mabye because of different background patterns.}
	\label{fig:5.18}
\end{figure}

\begin{figure}[htbp!]
	\centering
	\includegraphics[width=\linewidth]{figure/Fig_5.21.png}
	\caption{Temporal evolution of model reponse to input pattern from Fig.\ 5.21 \cite{zhaoping_understanding_2014}. Temporal evolution is faster than in the reference figure, which suggest a logical error in the model implementation.}
	\label{fig:5.21}
\end{figure}

\paragraph{Goals for coming week}
\begin{itemize}
    \item Finish reading and summarize chapter 5.8.2, i.e.\ 5.8.2.4 - 4.8.2.7 \cite{zhaoping_understanding_2014}
    \item Refactor existing code and implement the model in increasing complexity starting from two interacting pairs of neurons (see chapter 5.8.2 \cite{zhaoping_understanding_2014}), verify the implementation through analytic properties and replicate the findings of chapter 5.8.2. \cite{zhaoping_understanding_2014} up until I..
    \item ..find the cause of the difference between the results of the two different model implementation and the slow model dynamics of Fig.\ \ref{fig:5.21}
    \item To leave enough time to find a good project, brainstorm and talk about initial ideas for possible extensions with Prof.\ Zhaoping latest during next lab meeting
    \item If time permits, start replicating Fig.\ 5.34 and 5.35, i.e.\ the Figure-Ground Segmentation and Medial Axis Effect \cite{zhaoping_understanding_2014}
\end{itemize}

\section*{3. Week}

\paragraph{Review of Goals from last week}
I represented my results during the lab meeting.
\begin{itemize}
	\item finished reading chapter 5.8.2, i.e.\ 5.8.2.4 - 4.8.2.7 \cite{zhaoping_understanding_2014}
	\item refactored existing code and implemented two interacting pairs of neurons model (see chapter 5.8.2 \cite{zhaoping_understanding_2014}) replicating Fig.\ 5.58 \cite{zhaoping_understanding_2014} 
	\item found the cause of the difference between the results of the two different model implementations
	\item brainstorm and talked about initial ideas for possible extensions with Prof.\ Zhaoping latest during next lab meeting: try to find a new visual illusion with the V1 model based on deeper understanding of the dynamics than possibly herself
	\item started replicating Fig.\ 5.34 and 5.35, i.e.\ the Figure-Ground Segmentation and Medial Axis Effect \cite{zhaoping_understanding_2014}
\end{itemize}

\paragraph{Goals for coming week}
\begin{itemize}
	\item debug previous figures upon discussion: bigger image size for Fig.\ 5.21, scanning initial conditions for Fig. 5.58 \cite{zhaoping_understanding_2014}
	\item improve replication of Fig.\ 5.18 by averaging over random input and change bar width to make 5.18 F obvious \cite{zhaoping_understanding_2014} 
	\item finish replicating  Fig.\ 5.34 and 5.35, i.e.\ the Figure-Ground Segmentation and Medial Axis Effect \cite{zhaoping_understanding_2014}
	\item summarize and understand chapter 5.8.2, i.e.\ 5.8.2.4 - 4.8.2.7 \cite{zhaoping_understanding_2014}, assure you understand every detail in this chapter to think about possibly visual illusions
	\item replicate and understand visual search asymmetries, i.e.\ Fig.\ 5.31 \cite{zhaoping_understanding_2014}
	\item understand flow diagram from Vlad and help him getting feedback
\end{itemize}

\section*{4. Week}

\paragraph{Review of Goals from last week}
I represented my results during the lab meeting.
\begin{itemize}
	\item debugged previous figures upon discussion: bigger image size for Fig.\ 5.21, finding logical error in replicating Fig. 5.58 \cite{zhaoping_understanding_2014}
	\item improved replication of Fig.\ 5.18 by changing bar width to make 5.18 F obvious \cite{zhaoping_understanding_2014} 
	\item finished replicating  Fig.\ 5.34 and 5.35, i.e.\ the Figure-Ground Segmentation and Medial Axis Effect, including summation curve \cite{zhaoping_understanding_2014}
	\item therewhile discovered leaking-out problem and analyized filling-in by replicating Fig.\ 5.64.B and 5.61.A-B \cite{zhaoping_understanding_2014}
	\item summarized and understood chapter 5.8.2, i.e.\ 5.8.2.4 - 4.8.2.7 \cite{zhaoping_understanding_2014}
	\item started to understand flow diagram from Vlad and helped him getting feedback
\end{itemize}

\paragraph{Goals for coming week}
\begin{itemize}
	\item Finish replicating and understanding visual search asymmetry 
	\item Analyze and compare original and my V1 model implementation to
	find bug; then repeat replications with debugged model
	\item Read “Context effects on texture border localization bias” by Ariella 
	Popple or come up with other ideas to test
	\item Finish helping Vlad with and understand remaining flow diagram
	\item Start experimental preparation, at least with planning experimental 
	draft and ask / get feedback from Vlad and Fani
\end{itemize}

\section*{5. Week}
I will present my results in the next lab meeting.
\paragraph{Review of Goals from last week}
\begin{itemize}
	\item I read, summarized and explained “Context effects on texture border localization bias”, Popple (2003) in a presentation
	\item I came up with the idea of an entailed size illusion inverse to the well-known "Helmholtz illusion" (1867), raising an interesting puzzle to uncover with a mixture of modeling and experiment
	\item I clarified the problem statement by writing an abstract, got feedback from the lab and applied for a pre-data poster at the TeaP conference 2026 in Tübingen 
	\item I started to analyze and compare the original and my V1 model implementation for debugging; no logical differences in the implementation of connections, normalization or tuning curve
	\item I finished helping Vlad with and understood the remaining flow diagram
\end{itemize}

\paragraph{Goals for coming week}
\begin{itemize}
	\item Compare remaining model parts, possibly run original model, and find bug
	\item Plan experimental procedure in detail
	\item use PsychoPy to set up and build a first draft of the experiment, get fast feedback from the lab
	\item Take advantage of learning resources on how to do a psychophysics experiment
	\item Take time to systematically go through possible solutions of questions as well as to think about and test out less artifacted texture stimuli
\end{itemize}
	